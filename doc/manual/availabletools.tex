
\section{Available Tools} 

As already mentioned, ViennaMesh meshing tools are provided via C++-functors.
At the moment, we focus on unstructured mesh generation, adaptation, and 
classification methods based on simplicial mesh elements.

The following table provides an overview of the currently available tools.


\begin{table}[!ht]
\begin{center}
\begin{tabular}{|c|c|c|c|}
\hline
\textbf{Generation} & \textsc{Incremental Delaunay} & \textsc{Advancing Front} & \textsc{Raw} \\  
\hline
\textsc{2D}        &        1 & 0 & 0 \\
\textsc{32D}       &        0 & 0 & 1 \\
\textsc{3D}        &        1 & 1 & 0 \\
\hline
\end{tabular}
\end{center}
\caption{There is one two-dimensional mesh generator available (Triangle) and 
one hull mesh generator (CervPT). Note that CervPT is not based on a specific 
meshing algorithm, as it simply triangulates a given Polygon without any 
quality constraints. For three-dimensional volume meshing 
two meshing kernels are available, based on different meshing libraries (TetGen, Netgen).}
\end{table} 

\begin{table}[!ht]
\begin{center}
\begin{tabular}{|c|c|c|c|}
\hline
\textbf{Adaptation} & \textsc{Orientation} & \textsc{Quality Improvement} & \textsc{Cell Normals} \\  
\hline
\textsc{2D}         &        0 &        0 &        0 \\
\textsc{32D}        &        1 &        1 &        1 \\
\textsc{3D}         &        0 &        0 &        0 \\
\hline
\end{tabular}
\end{center}
\caption{At this stage of development the primary focus is on hull meshes. Therefore, 
the mesh adaptation tools are only available for these types of meshes, ultimately 
providing a statistical overview of the quality of the whole mesh.}
\end{table} 

\begin{table}[!ht]
\begin{center}
\begin{tabular}{|c|c|}
\hline
\textbf{Classification} & \textsc{Cell Aspect Ratio}\\  
\hline
\textsc{2D}         &        0 \\
\textsc{32D}        &        0 \\
\textsc{3D}         &        1 \\
\hline
\end{tabular}
\end{center}
\caption{A mesh classification tool is available which analyzes the quality of 
each mesh element of a simplicial volume mesh.}
\end{table} 
