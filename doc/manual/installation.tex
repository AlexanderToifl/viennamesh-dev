\section{Installation}

By default, executables of all of the utilities are available in the 

\begin{exaipd}
\begin{Verbatim}
bin/ 
\end{Verbatim}
\end{exaipd}

folder. The executables are statically linked to the used external libraries. 
Therefore they should be ready to use.
However, if ViennaMesh has to be rebuild, this section provides the required 
informations.

% -----------------------------------------------------------------------------
% -----------------------------------------------------------------------------
\subsection{Folder Hierarchy}
% -----------------------------------------------------------------------------
% -----------------------------------------------------------------------------
The package is modularized, meaning, that the tools are available as 
distinct applications. Therefore further development can be specifically 
oriented. In the following, the folder hierarchy is shown.

\begin{exaipd}
\begin{Verbatim}
bin/                    :: this folder contains all built executables
doc/                    :: the manual sources are here
examples/               :: an example input file is provided
hull_adaptor/           :: source files for the hull mesh adaption tool
hull_converter/         :: source files for the mesh conversion utility
hull_orienter/          :: source files for a simple orientation repair tool
mesh_classifier/        :: source files for a mesh quality evaluation utility
volume_mesher/          :: source files for the volume mesher
LICENSE                 :: the LGPL license file
build_all.sh            :: script which builds all tools 
clean_all.sh            :: script which removes all build related files
\end{Verbatim}
\end{exaipd}

% -----------------------------------------------------------------------------
% -----------------------------------------------------------------------------
\subsection{Dependencies}
% -----------------------------------------------------------------------------
% -----------------------------------------------------------------------------
The whole package comes with almost all dependent software libraries.
However, it requires the Boost libraries~\cite{boost} to be present.
By default, the build environment expects the boost libraries to be present 
in the system path
\begin{exaipd}
\begin{Verbatim}
/usr/include
\end{Verbatim}
\end{exaipd}

If the Boost libraries are present in a location different to the system 
path, it can be mentioned at the build scripts at the different 
utilities. For example, the build script of the volume mesher looks like this:

\begin{exaipd}
\begin{Verbatim}
#!/bin/bash

# configure waf
#
./waf configure 

# build
#   note: waf uses automatically all available cores
./waf --progress

# strip executable and copy to bin folder
#
strip build/volume_mesher
cp build/volume_mesher ../bin/
\end{Verbatim}
\end{exaipd}

To specify a Boost path, the configuration step has to be altered:

\begin{exaipd}
\begin{Verbatim}
# configure waf
#
./waf configure --boost-includes=/path/to/boost/include
\end{Verbatim}
\end{exaipd}


% -----------------------------------------------------------------------------
% -----------------------------------------------------------------------------
\subsection{Building}
% -----------------------------------------------------------------------------
% -----------------------------------------------------------------------------
A script is provided which builds all available executables.
Executing the script 

\begin{exaipd}
\begin{Verbatim}
./build_all.sh
\end{Verbatim}
\end{exaipd}

builds all utilities and moves the executables to the 

\begin{exaipd}
\begin{Verbatim}
bin/
\end{Verbatim}
\end{exaipd}

folder.

Note, that the build system is based on Waf~\cite{waf}.

%This chapter shows how {\ViennaCL} can be integrated into a project and how the
%examples are built. The necessary steps are outlined for several different
%platforms, but we could not check every possible combination of hardware,
%operating system and compiler. If you experience any trouble, please write to
%the maining list at \\
%\begin{center}
%\texttt{viennacl-support$@$lists.sourceforge.net} 
%\end{center}


%% -----------------------------------------------------------------------------
%% -----------------------------------------------------------------------------
%\section{Dependencies}
%% -----------------------------------------------------------------------------
%% -----------------------------------------------------------------------------
%\label{dependencies}
%{\ViennaCL} uses the {\CMake} build system for multi-platform support.
%Thus, before you proceed with the installation of {\ViennaCL}, make sure you
%have a recent version of {\CMake} installed.

%\begin{itemize}
% \item A recent C++ compiler (e.g.~{\GCC} version 4.2.x or above and Visual C++
%2008 are known to work)
% \item {\OpenCL}~\cite{khronoscl,nvidiacl} for accessing compute devices (GPUs);
%see Section~\ref{opencllibs} for details.
%(optional, since iterative solvers can also be used with e.g.~{\ublas})
% \item {\CMake}~\cite{cmake} as build system (optional, but highly recommended
%for building the examples)
%% \item {\OpenMP}~\cite{openmp} for the benchmark suite (optional, but
%%recommended for fair benchmarking)
% \item {\ublas} (shipped with {\Boost}~\cite{boost}) provides the same interface as {\ViennaCL} and allows to switch between CPU and GPU seamlessly, see the tutorials.
%\end{itemize}

%%The use of {\OpenMP} for the benchmark suite allows fair comparisons between your multi-core CPU and your compute device (e.g.~GPU).

%\section{Generic Installation of ViennaCL} \label{sec:viennacl-installation}
%Since {\ViennaCL} is a header-only library, it is sufficient to copy the folder
%\lstinline|viennacl/| either into your project folder or to your global system
%include path. On Unix based systems, this is often \lstinline|/usr/include/| or
%\lstinline|/usr/local/include/|.

%On Windows, the situation strongly depends on your development environment. We advise users
%to consult the documentation of their compiler on how to set the include
%path correctly. With Visual Studio this is usually something like \texttt{C:$\setminus$Program Files$\setminus$Microsoft Visual Studio 9.0$\setminus$VC$\setminus$include}
%and can be set in \texttt{Tools -> Options -> Projects and Solutions -> VC++-\-Directories}. The include and library directories of your {\OpenCL} SDK should also be added there.

%\NOTE{If multiple {\OpenCL} libraries are available on the host system, one has
%to ensure that the intended one is used.}


%% -----------------------------------------------------------------------------
%% -----------------------------------------------------------------------------
%\section{Get the {\OpenCL} Library}
%\label{opencllibs}
%% -----------------------------------------------------------------------------
%% -----------------------------------------------------------------------------
%The development of {\OpenCL} applications requires a corresponding library
%(e.g.~\texttt{libOpenCL.so} under Unix based systems) and a suitable driver if
%used on GPUs. This section describes how these can be acquired.

%\TIP{Note, that for Mac OS X systems there is no need to install an {\OpenCL} 
%capable driver and the corresponding library. 
%The {\OpenCL} library is already present if a suitable graphics 
%card is present. The setup of {\ViennaCL} on Mac OS X is discussed in
%Section~\ref{apple}.}

%\subsection{\NVIDIA Driver}
%\NVIDIA provides the {\OpenCL} library with the GPU driver. Therefore, if a 
%\NVIDIA driver is present on the system, the library is too. However, 
%not all of the released drivers contain the {\OpenCL} library. 
%A driver which is known to support {\OpenCL}, and hence providing the required
%library, is $195.36.24$. 

%\subsection{ATI Stream SDK} \label{sec:opencl-on-ati}
%ATI provides the {\OpenCL} library with the Stream
%SDK~\cite{atistream}. At the release of {\ViennaCLversion}, the latest Stream
%SDK was $2.1$. If used with ATI GPUs, the ATI GPU drivers of version at least
%$10.4$\footnote{Current ATI drivers
%may not work with current kernel version. The presented tests are based on the 
%$2.6.33$ kernel.} are required. If {\ViennaCL} is to be run on multi-core CPUs,
%no additional GPU driver is required. The installation notes
%of the SDK provides guidance throughout the
%installation process~\cite{atistreamdocu}. 

%\TIP{If the SDK is installed in a non-system wide location, the 
%include and library paths have to be specified in the
%\texttt{CMakeLists.txt} files. An example of how to set these paths is
%provided in the file \texttt{CMakeLists.txt} around line 18. Be sure to add the
%{\OpenCL} library path to the \texttt{LD\_LIBRARY\_PATH} environment variable.
%Otherwise, linker errors will occur as the required library cannot be found.}

%It is important to note that the ATI Stream SDK does not provide full 
%double precision support~\cite{atidouble} on CPUs and GPUs, so it is only experimentally
%available in {\ViennaCL}. This experimental mode is disabled by default, but can
%be enabled for either CPUs or GPUs by defining one of the preprocessor constants
%\begin{lstlisting}
%// for CPUs:
%#define VIENNACL_EXPERIMENTAL_DOUBLE_PRECISION_WITH_STREAM_SDK_ON_CPU
%// for GPUs:
%#define VIENNACL_EXPERIMENTAL_DOUBLE_PRECISION_WITH_STREAM_SDK_ON_GPU
%\end{lstlisting}
%prior to any inclusion of {\ViennaCL} header files.

%\NOTE{Some compute kernels may not work as expected in the experimental double
%precision mode in the ATI Stream SDK. Moreover, the functions \texttt{norm\_1}, \texttt{norm\_2}, \texttt{norm\_inf} and \texttt{index\_norm\_inf} are not available on GPUs in double precision using ATI Stream SDK.}

%% -----------------------------------------------------------------------------
%% -----------------------------------------------------------------------------
%\section{Building the Examples and Tutorials}
%% -----------------------------------------------------------------------------
%% -----------------------------------------------------------------------------
%For building the examples, we suppose that {\CMake} is properly set up
%on your system. The other dependencies are listed in Tab.~\ref{tab:tutorial-dependencies}.

%\begin{table}[tb]
%\begin{center}
%\begin{tabular}{l|l}
%Tutorial No. & Dependencies\\
%\hline
%\texttt{tutorial/tut1.cpp}      & {\OpenCL} \\
%\texttt{tutorial/tut2.cpp}      & {\OpenCL}, {\ublas} \\
%\texttt{tutorial/tut3.cpp}      & {\OpenCL}, {\ublas} \\
%\texttt{tutorial/tut4.cpp}      & {\ublas} \\
%\texttt{tutorial/tut5.cpp}      & {\OpenCL} \\
%\texttt{benchmarks/vector.cpp}  & {\OpenCL} \\
%\texttt{benchmarks/sparse.cpp}  & {\OpenCL}, {\ublas} \\
%\texttt{benchmarks/solver.cpp}  & {\OpenCL}, {\ublas} \\
%\end{tabular}
%\caption{Dependencies for the examples in the \texttt{examples/} folder}
%\label{tab:tutorial-dependencies}
%\end{center}
%\end{table}

%\subsection{Linux}
%To build the examples, open a terminal and change to:

%\begin{exaipd}
%\begin{Verbatim}
%$> cd /your-ViennaCL-path/build/
%\end{Verbatim}
%\end{exaipd}

%Execute

%\begin{exaipd}
%\begin{Verbatim}
%$> cmake ..
%\end{Verbatim}
%\end{exaipd}

%to obtain a Makefile.
%Executing

%\begin{exaipd}
%\begin{Verbatim}
%$> make 
%\end{Verbatim}
%\end{exaipd}

%builds the examples. If some of the dependencies in Tab.~\ref{tab:tutorial-dependencies} are not fulfilled, you can build each example separately:
%\begin{exaipd}
%\begin{Verbatim}
%$> make tut1              #builds tutorial 1
%$> make vectorbench       #builds vector benchmarks
%\end{Verbatim}
%\end{exaipd}


%\TIP{Speed up the building process by using jobs, e.g. \keyword{make -j4}.}

%\subsection{Mac OS X}
%\label{apple}
%The tools mentioned in Section \ref{dependencies} are available on 
%macintosh platforms too. 
%For the {\GCC} compiler the Xcode~\cite{xcode} package has to be installed.
%To install {\CMake} and {\Boost} external portation tools have to be used, 
%for example, Fink~\cite{fink}, DarwinPorts~\cite{darwinports} 
%or MacPorts~\cite{macports}. Such portation tools provide the 
%aforementioned packages, {\CMake} and {\Boost}, for macintosh platforms. 

%\TIP{If the {\CMake} build system has problems detecting your {\Boost} libraries, 
%determine the location of your {\Boost} folder. 
%Open the \texttt{CMakeLists.txt} file in the root directory of {\ViennaCL} and 
%add your {\Boost} path after the following entry: 
%\texttt{IF(\${CMAKE\_SYSTEM\_NAME} MATCHES "Darwin")} }

%The build process of {\ViennaCL} is similar to Linux.

%\subsection{Windows}
%In the following the procedure is outlined for \texttt{Visual Studio}: Assuming that an {\OpenCL} SDK and {\CMake} is already installed, Visual Studio solution and project files can be created using {\CMake}:
%\begin{itemize}
%\item Open the {\CMake} GUI.
%\item Set the {\ViennaCL} base directory as source directory.
%\item Set the \texttt{build/} directory as build directory.
%\item Click on 'Configure' and select the appropriate generator (e.g.~\texttt{Visual Studio 9 2008})
%\item Click on 'Generate' (you may need to click on 'Configure' one more time before you can click on 'Generate')
%\item The project files can now be found in the {\ViennaCL} build directory, where they can be opened and compiled with Visual Studio (provided that the include and library paths are set correctly, see Sec.~\ref{sec:viennacl-installation}).
%\end{itemize}

%\TIP{The examples and tutorials should be executed from within the \texttt{build/} directory of {\ViennaCL}, otherwise the sample data files cannot be found.}

























