
\section{Introduction}  % \addcontentsline{toc}{chapter}{Introduction}

Due to increasing computational power with a vast performance even in desktop systems, new possibilities for multiphysics simulations arise, which intensify the tension placed on mesh generation. On the one hand, this results in an increased requirement for the robustness of the algorithms employed, while on the other hand, it also forces the use of high performance methodologies in order to reduce meshing times. 
The quality of the mesh is not only critically important to the quality of the calculated results: failure to properly control the meshing process can also jeopardize or even completely prevent simulation. Since meshing is the first initial step of the simulation flow, all subsequent results depend on this fundamental step. 

A parallel meshing and adaptation approach, combining Delaunay and advancing front algorithms suitable for finite volume and finite element discretization schemes for three dimensions, has therefore been developed~\cite{stimpflrobust,stimpflperf,stimpflmulti,heinzlgen}. 

Parallelization and the robustness of the algorithms are facilitated by employing a rigorous surface treatment, which not only enforces the prescribed quality criteria such as the Delaunay property, but also allows the decoupling of the subsequent parallel volume meshing steps. This decoupling is of fundamental importance to the volume meshing step. If the surface remains consistent, there is no need to exchange any data between the individual threads of the volume meshing process. This also enables the inherent utilization of many-core CPUs. Parallelization is implemented by suitably combining multiple programming paradigms and following modern design guidelines, which is necessary in order to keep the development on multi-core processors as simple as possible, while not forsaking any of their computational power. 
The use of multi-core processors can drastically reduce the mesh generation time as well as the time for execution of subsequent modules, e.g., a linear solver. The availability of high quality and robust high performance mesh generation tools is therefore of utmost importance.


