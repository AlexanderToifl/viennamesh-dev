\chapter{Pitfalls} \label{chap:pitfalls}
The increased reusability of classes due to a separation of intrinsic and associated data (cf.~\hyperref[intro]{Introduction})
introduces a few pitfalls one should be aware of. Depending on the code design and the application,
these pitfalls can be somewhere between completely unrealistic and painful. In the following, we discuss the most important of them.

\section{Object Lifetimes}
Consider the following code, where the class \lstinline|A| is defined somewhere else:
\begin{lstlisting}
 A * myA = new A();
 viennadata::access<long, double>(42)(*myA) = 3.1415;
 delete myA;
\end{lstlisting}
The problem here is that data associated with \lstinline|A| is not erased prior to the deletion of \lstinline|myA|.
Thus, there is still the double value $3.1415$ stored using the address of \lstinline|myA|, but there is no direct way to access that data.
To avoid confusion: The data is not lost in a sense of memory leak, because the tree managing the data is still consistent and properly free'd at program exit.
However, the user has no direct way to retain an object which has the same address as the one pointed to by \lstinline|myA|.

In order to ensure that all data is properly deleted, when the respective object is destroyed, the following options are available:
\begin{itemize}
 \item Manually call \lstinline|viennadata::erase<KeyType, DataType>()(*myA)| for all pairs of \lstinline|KeyType| and \lstinline|DataType| used.
 \item Manually ensure that \lstinline|viennadata::erase<all,all>()(*myA)| is called prior to deleting \lstinline|*myA|. This requires that all data is registered properly, which is by default the case.
 \item Use \lstinline|viennadata::free(*myA)| instead of \lstinline|delete|. This will call \lstinline|viennadata::erase<all,all>()(*myA)| and then \lstinline|delete myA|. 
 It will not work, if \lstinline|myA| points to an array, for which \lstinline|delete[]| has be called.
 \item If the sources of \lstinline|myA| can be modified, then one can inject the call to \lstinline|viennadata::erase<all,all>()(*myA)| into the destructor as follows:
\begin{lstlisting}
 A::~A()
 {
   viennadata::erase<all,all>()(*this);
 }
\end{lstlisting}
This will automatically remove all data stored for the object, unless automatic registration of key and data types is turned off.
 \item Provide another identification mechanism that is not based on object lifetimes, see Section \ref{sec:id-retrieval}. Thus, if an object with the same ID is created at some later time, the data can be accessed again.
\end{itemize}


\section{Polymorphism}
Since {\ViennaData} relies on a static type system, it cannot reasonably deal with run time polymorphism. Consider
\begin{lstlisting}
 class Base { };
 class Derived : public Base {};

 Derived d;
 Base * pb = &d;
 viennadata<long, double>(360)(*pb) = 3.1415;
 std::cout << viennadata<long, double>(360)(d) << std::endl;
\end{lstlisting}
Even though data is stored for the same object, the resulting program will still print a zero value. 
The reason is that data is stored for an object of type \lstinline|Base|, while it is retrieved for an object of type \lstinline|Derived|.

%At present there is no direct way to overcome this issue. Future versions of {\ViennaData} may provide means to work with pointers to base classes.
Better support for run time polymorphism might be provided in future versions of {\ViennaData}. In the current version, library users are advised to use a custom identification mechanisms and provide objects of the base type only.

\NOTE{The use of {\ViennaData} with run time polymorphism may lead to unexpected results if no custom identification mechanism is provided!}

